\documentclass[letterpaper, 10pt, DIV=13]{scrartcl}
\usepackage[T1]{fontenc}
\usepackage[english]{babel}
\usepackage{amsmath, amsfonts, amsthm, xfrac}

\numberwithin{equation}{section}
\numberwithin{figure}{section}
\numberwithin{table}{section}

\usepackage{sectsty}
\allsectionsfont{\normalfont\scshape} % Make all section titles in default font and small caps.

\usepackage{fancyhdr} % Custom headers and footers
\pagestyle{fancyplain} % Makes all pages in the document conform to the custom headers and footers

\fancyhead{} % No page header - if you want one, create it in the same way as the footers below
\fancyfoot[L]{} % Empty left footer
\fancyfoot[C]{} % Empty center footer
\fancyfoot[R]{\thepage} % Page numbering for right footer

\renewcommand{\headrulewidth}{0pt} % Remove header underlines
\renewcommand{\footrulewidth}{0pt} % Remove footer underlines
\setlength{\headheight}{13.6pt} % Customize the height of the header

\setlength\parindent{0pt}
\pagenumbering{gobble}

\title {
	\normalfont
	\huge{Programming in the Past} \\
	\vspace{10pt}
	\large{CMPT 331 - Spring 2023 | Dr. Labouseur}
}

\author{\normalfont Josh Seligman | joshua.seligman1@marist.edu}

\pagenumbering{arabic}
\begin{document}
\maketitle

\section{Log}
\subsection{Predicton}
I am predicting that it will take me around \textbf{20 hours (average of 4 hours per programming language)} for me to learn Fortran, COBOL, BASIC, Pascal, and Scala and build a Caesar cipher in each language. This is an extremely rough estimate as I have never used any of these programming languages before and will have to learn each of them starting with "hello world." I am sure that I will have moments of staring at my computer screen for extended periods of time due to a weird nuance or gimmick in at least one of these programming languages that is not common in more modern languagues. However, despite my lack of familiarity with these languages, I am hoping that there will be some nice similarities between them, so my approach to writitng the Caesar cipher does not drastically change between them, and each implementation can easily be compared to each other on an even playing field.

\subsection{Progress Log}
\begin{center}
	\begin{tabular}{|p{1in}|p{1in}|p{4in}|}
		\hline
		Date & Hours Spent & Tasks / Accomplishments / Issues / Thoughts \\
		\hline
		January 17 & 0 hours & Lorem ipsum dolor sit amet, consectetur adipiscing elit. Nulla fringilla arcu quis feugiat cursus. Morbi a arcu vitae erat consectetur rhoncus nec ac augue. Nunc fringilla, diam a pellentesque feugiat, augue magna porta justo, lacinia viverra neque lacus non odio. Curabitur vehicula odio nec hendrerit ullamcorper. Phasellus elementum posuere libero, non efficitur nisl vulputate id. Pellentesque tincidunt nunc nunc, sed vehicula justo tempor nec. Lorem ipsum dolor sit amet, consectetur adipiscing elit. Vivamus non sapien at sem eleifend sollicitudin eget id nibh. Morbi a nulla nec velit sollicitudin pretium et ut lorem. Etiam sit amet mollis velit, non convallis libero. Phasellus vestibulum euismod ante. Nulla sem elit, mattis a interdum vel, maximus porta purus. Phasellus ac facilisis mauris. Phasellus vulputate magna et lectus tempor, vel venenatis arcu iaculis. Mauris facilisis varius odio non imperdiet. Morbi eu euismod orci.  \\
		\hline
	\end{tabular}
\end{center}

\subsection{Final Results and Analysis}

\section{Commentary}
\subsection{Fortran}

\subsection{COBOL}

\subsection{BASIC}

\subsection{Pascal}

\subsection{Procedural Scala}

\section{Conclusion}

\end{document}